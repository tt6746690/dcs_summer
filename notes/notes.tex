\documentclass[11pt]{article}
\input{\string~/.preamble}

\begin{document}

% arg1=pdfurl arg2=pagenum arg3=sectiontitle
\newcommand{\linksection}[3][numerical_methods_in_finance_and_economics_matlab_intro.pdf]{
    \subsection*{\href[page=#2]{#1}{#3}}
}

\renewcommand{\norm}[1]{\left\lVert#1\right\rVert}
\renewcommand{\E}[2][]{\mathbb{E}_{#1}\left\{#2\right\}}
\newcommand{\var}[2][]{var_{#1}\left\{#2\right\}}
\newcommand{\cov}[1]{cov\{#1\}} 
\newcommand{\normal}[1]{\mathcal{N}\left(#1\right)}
\newcommand{\exponents}[1]{exp\left\{#1\right\}}

\newcommand{\bmu}{\boldsymbol{\mu}}
\newcommand{\bpi}{\boldsymbol{\pi}}
\newcommand{\bTheta}{\boldsymbol{\Theta}}
\newcommand{\bSigma}{\boldsymbol{\Sigma}}
\newcommand{\bphi}{\boldsymbol{\phi}}

\newcommand{\calL}{\mathcal{L}}
\newcommand{\calE}{\mathcal{E}}
\newcommand{\calR}{\mathcal{R}}
\newcommand{\calC}{\mathcal{C}}
\newcommand{\calD}{\mathcal{D}}
\newcommand{\bx}{\matr{x}}
\newcommand{\bt}{\matr{t}}
\newcommand{\bw}{\matr{w}}
\newcommand{\bX}{\matr{X}}
\newcommand{\bZ}{\matr{Z}}
\newcommand{\bz}{\matr{z}}
\newcommand{\bu}{\matr{u}}


 

\linksection{28}{1 Motivation}

 
\begin{enumerate}
    \item \textbf{derivatives} financial assets that derives value from the performance of an underlying, e.x. asset, index, interest rate
    \item \textbf{options} the buyer has the option to sell/buy an underlying asset; the seller has the obligation to fullfill such request. The buyer pays the seller a premium for this right
    \item \textbf{Black–Scholes–Merton model} gives theoretical estimates of European-style options and shows that the option has a unique price regardless of the risk of the security and its expected return    
\end{enumerate} 
 

\linksection{48}{2 Financial Theory}
 
\begin{enumerate}
    \item \textbf{Uncertainty} probability distribution of a random variable, i.e. price of stock, captures the uncertainty. Capture subjective uncertainty by updating the posterior. Binomial model are building blocks for discrete models; Wiener process with differential equation are buildingblocks for continuous models
    \item \textbf{Financial assets}
    \begin{enumerate}
        \item \textbf{bond} firms/administration issues bond to fund their activities. Bond does not imply ownership of a firm. The buyer of bonds lend issuer some money. At maturity, issuer pay the bond owner an ammount called par value and periodically as coupons
        \item \textbf{stocks} entitle owner to a share of the issuing firm. stocks are limited liability assets, i.e. buyer not responsible for undesirable conduct of the firm. Stocks havce no predefined maturity and entitle to owner to some stream of payments by nature of dividents, which are stochastic, i.e. depend on how the firm is faring. 
        \item \textbf{derivatives} financial contracts, dependent on value of some underlying variable, i.e. stock price, interest rate, index, non-financial asset.
    \end{enumerate}
\end{enumerate}
 

\linksection{234}{4 Numerical Integration: Deterministic and Monte Carlo Methods}


\linksection{236}{4.1 Determinstic Quadrature}

\begin{enumerate}
    \item \textbf{Goal} approximate the value of a definite integral 
    \[
        I(f) = \int_{a}^b f(x) dx    
    \]
    \item \textbf{Idea} Approximate the integral by a weighted sum of integrand evaluated at a set of integration points 
    \[
        Q(f) = \sum_j w_j f(x_j)  
    \]
    for $a = x_0 < x_1 < \cdots < x_N = b$ 
\end{enumerate}



\linksection{236}{4.2 Monte Carlo Integration}

\begin{enumerate}
    \item \textbf{Idea} Cast determinstic evaluation of integral to evaluation of expected value of a random variable. To evaluate 
    \[
        I = \int_0^1 g(x) dx 
        \qquad 
        \longrightarrow 
        \qquad 
        I = \E{g(U)}
        \quad U\sim Unif(0,1)
    \]
    We can estimate the expected value by generating a sequence $\{U\}_i$ independent random samples from uniform distribution and evaluate 
    \[
        \hat{I}_m = \frac{1}{m} \sum_{i=1}^m g(U_i)
    \]
    where $\lim_{m\to \infty} \hat{I}_m = I$ with probability 1 by all of large numbers
\end{enumerate}



\end{document}
